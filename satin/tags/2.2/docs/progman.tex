\documentclass[10pt]{article}

\usepackage{url}
\usepackage{graphicx}
\usepackage{times}
\usepackage{color}
\usepackage{floatpag}
\newcommand{\todo}[1]{\textbf{\textcolor{red}{[TODO: #1]}}}
\usepackage{listings}
\usepackage{courier} % For a richer set of \tt fonts (for listings)
\lstset{columns=fullflexible,showstringspaces=false,numberstyle=\tiny,basicstyle=\footnotesize,breaklines=false}
%\lstset{basicstyle=\ttfamily\footnotesize,columns=fullflexible,showstringspaces=false,numberstyle=\tiny,breaklines=false}

\begin{document}

\title{The Satin Divide-and-Conquer System Programmer's Manual}

\author{The Ibis Group}

\maketitle

\section{Introduction}

Satin is a parallel programming environment for divide-and-conquer
parallelization, and master-worker parallelization. Satin extends Java
with two simple primitives for divide-and-conquer programming: \texttt{spawn}
and \texttt{sync}. The Satin byte-code rewriter and runtime system cooperate to
implement these primitives efficiently on top of the IPL.  

This manual explains how to program applications using Satin.
For information on how to run applications using Satin, see the Satin
user's manual, available in the Satin distribution in the
\texttt{docs} directory.

\section{Satin jobs}

To use Satin, the programmer must label one or more methods in his
class as Satin jobs. This is done by defining an interface that
extends the Satin interface ibis.satin.Spawnable. For example:

{\small
\begin{quote}
\begin{verbatim}
package search;

interface Searcher extends ibis.satin.Spawnable {
   public int search(int a[], int from, int to, int val);
}
\end{verbatim}
\end{quote}
}
\noindent

All methods that implement Searcher.search() will be Satin jobs, they
are marked as spawnable.  In general such a marker interface may
contain an arbitrary number of methods.  When a Satin program is
compiled with the Satin compiler, methods marked as spawnable may be
executed in parallel. A class that has spawnable methods 
must extend the special class ibis.satin.SatinObject.
The result of a Satin job can only be
used after the invocation of the sync method, which lives in
ibis.satin.SatinObject. The sync method is guaranteed to only
terminate after the earlier job invocations have terminated.  Satin
imposes one restriction on the type of parameters and return type of a
Satin job: they must be of a basic type or must be serializable.
Since a SatinObject is serializable, all its subclasses are
serializable as well. This restriction ensures that a Satin job can be
stored and moved from one processor to another. Note that this means that
all fields of a Satin object must be either serializable or transient.

\begin{figure}[t!]
{\small
\begin{verbatim}
package search;

import ibis.satin.SatinObject;

class SearchImpl1 extends SatinObject implements Searcher {
   public int search(int a[], int from, int to, int val) {
      for(int i = from; i < to; i++) {
         if (a[i] == val) return i;
      }
      return -1;
   }

   public static void main(String[] args) {
      SearchImpl1 s = new SearchImpl1();
      int a[] = new int[200];

      // Fill the array with random values between 0 and 100.
      for(int i=0; i<200; i++) {
         a[i] = (int) (Math.random() * 100);
      }

      // Search for 42 in two sub-domains of the array.
      // Because the search method is marked as spawnable,
      // Satin can run these methods in parallel.
      int res1 = s.search(a, 0, 100, 42);
      int res2 = s.search(a, 100, 200, 42);

      // Wait for results of the two invocations above.
      s.sync();

      // Now compute an overall result.
      int res = (res1 >= 0) ? res1 : res2;

      if(res >= 0) {
         System.out.println("found at pos: " + res);
      } else {
         System.out.println("element not found");
      }
   }
}
\end{verbatim}
}
\caption{Parallel search with Satin.}
\label{satin-search-fig}
\end{figure}

As an example, Figure~\ref{satin-search-fig} shows an implementation of the search
method. It just looks at a range of elements of the passed array, and
tries if one of the elements equals val. If so, it returns the index
of the element. If no matching element was found, -1 is returned.
Now, we can invoke search as is shown in the main method of Figure~\ref{satin-search-fig}.
In this case, we call search twice, once for the first half of the
array, and once for the second half. If we compile the program with
javac, the two search invocations will be done sequentially. The sync
method in SatinObject does nothing. So we can run the program normally
on any JVM.  If we modify the class files using the Satin bytecode
rewriter, however, the program will be converted to a parallel
program.  The two calls to search in the main class of Figure~\ref{satin-search-fig} will
be executed in parallel if two JVMs are available. To run the code on
more JVMs, the array can be split into more chunks. In general, an
arbitrary number of invocations to Satin job methods may be done.

\begin{figure}[t!]
{\small
\begin{verbatim}
import ibis.satin.SatinObject;

class SearchImpl2 extends SatinObject implements Searcher {
   public int search(int a[], int from, int to, int val) {
      if (from == to) { // The complete array has been searched.
          return -1;    // The element was not found.
      }
      if (to - from == 1) { // Only one element left.
          return (a[from] == val) ? from : -1; // It might be the one.
      }

      // Now, split the array in two parts and search them in parallel.
      int mid = (from + to) / 2;
      int res1 = search(a, from, mid, val);
      int res2 = search(a, mid, to, val);
      sync();
      return (res1 >= 0) ? res1 : res2;
   }

   // main method as in previous figure, except that it creates
   // a SearchImpl2 object instead of a SearchImpl1 object.
}
\end{verbatim}
}
\caption{Divide-and-conquer parallel search with Satin.}
\label{satin-par-search-fig}
\end{figure}

It is also allowed to recursively invoke job methods from job methods,
as is shown in Figure~\ref{satin-par-search-fig}. Here the search is recursively divided into
smaller problems until a problem remains that is trivially
handled. This model of programming is called divide-and-conquer. Satin
has special grid-aware load-balancing algorithms built-in to run such
programs efficiently on the grid, on any number of machines. This way,
the entire search can be done in parallel on a large number of
machines. The example shows how easy it is to write a real parallel
grid application using Satin.  

This Satin release includes a program that may help the programmer in determining
where to place the sync method invocations. The user manual gives more information
on how to use it.

\section{Exceptions and abort}

A Satin job may
throw exceptions in the usual way. This can be used to rapidly
terminate a large set of jobs, e.g., when a search result was
found. Terminating jobs that are no longer useful can be done with the
abort method from the SatinObject class.  Satin has its own exception
type: \texttt{ibis.satin.Inlet}.  By extending the \texttt{Inlet} class,
the programmer can inhibit the generation of a stack trace,
which is an expensive operation.
The stack trace is usually not useful in the Satin context,
because the exception is not an error, it is used to steer the
search. Extending \texttt{Inlet} is optional, regular exceptions can also be
used. Figures~\ref{satin-spec-search-fig1} and~\ref{satin-spec-search-fig2} change the search algorithm we used in the previous
examples to use exceptions.

\begin{figure}[t!]
{\small
\begin{verbatim}
package search;

import ibis.satin.SatinObject;
import ibis.satin.Inlet;

class SearchResultFound extends Inlet {
   int pos;

   SearchResultFound(int pos) {
      this.pos = pos;
   }
}

interface Searcher3 extends ibis.satin.Spawnable {
   public void search(int a[], int from, int to, int val)
      throws SearchResultFound;
}

class SearchImpl3 extends SatinObject implements Searcher3 {
   public void search(int a[], int from, int to, int val)
         throws SearchResultFound {

      if (from == to) { // The complete array has been searched.
         return;        // The element was not found.
      }
      if (to - from == 1) {    // Only one element left.
         if (a[from] == val) { // Found it!
            throw new SearchResultFound(from);
         } else {
            return; // The element was not found.
         }
      }

      // Now, split the array in two parts and search them in parallel.
      int mid = (from + to) / 2;
      search(a, from, mid, val);
      search(a, mid, to, val);
      sync();
   }

   ...
\end{verbatim}
}
\caption{Speculative parallel search with Satin.}
\label{satin-spec-search-fig1}
\end{figure}

\begin{figure}[t!]
{\small
\begin{verbatim}
   ...

   public static void main(String[] args) {
      SearchImpl3 s = new SearchImpl3();
      int a[] = new int[200];
      int res = -1;

      // Fill the array with random values between 0 and 100.
      for(int i=0; i<200; i++) {
         a[i] = (int) (Math.random() * 100);
      }

      // Search for 42 in two sub-domains of the array.
      // Because the search method is marked as spawnable,
      // Satin can run these methods in parallel.
      try {
         s.search(a, 0, 100, 42);
         s.search(a, 100, 200, 42);

         // Wait for results of the invocations above.
         s.sync();
      } catch (SearchResultFound x) {
         // We come here only if one of the two jobs found a result.
         s.abort(); // kill the other job that might still be running.
         res = x.pos;
         return; // return needed because inlet is handled in separate thread.
      }

      if(res >= 0) {
         System.out.println("found at pos: " + res);
      } else {
         System.out.println("element not found");
      }
   }
}
\end{verbatim}
}
\caption{Speculative parallel search with Satin.}
\label{satin-spec-search-fig2}
\end{figure}

Note that in this version the search method doesn't return a value. It
will throw a SearchResultFound object containing the result if the
element was found, or it will terminate without throwing an exception
if no result is found. This is also the reason the search method
implements the Searcher3 interface instead of the Seacher
interface. The declaration of the exception in the throws clause and
the return type are the only differences.  At the top level this
exception is caught, and the other search jobs can then be
terminated. They have become useless, because the element has already
been found in the array. The parallel search can thus be stopped. We
call the catch block in the main method that handles the result of the
search an inlet.

The way of searching in Figures~\ref{satin-spec-search-fig1} and~\ref{satin-spec-search-fig2} is called
speculative parallelism: the search speculatively starts on both two
parts of the array simultaneously, even though we know from the start
that a part of the search may be terminated as soon as the element is
found. We might thus do more work than a sequential search. On the
other hand, we might also do less work than the sequential version,
for instance if the element is in the beginning of the second part of
the array.

The inlet in main contains a return statement. This must
be there, because the inlet runs in a new thread. The original main
thread is still blocked in the sync statement. When the inlet returns,
the main thread is unblocked, because both search jobs have finished:
one threw an exception, the other has been aborted.  

\section{Satin job scheduling}

Satin offers a choice of three different job scheduling strategies:
\begin{description}
\item{Random work-stealing}.
When looking for work, each Satin instance
first examines its own job queue. When this job queue is empty,
it randomly selects another Satin instance and takes the first job on its job queue.
If that job queue is empty, it randomly selects another Satin instance, et cetera,
et cetera.  This strategy is the default, and has in the past been proven
to be optimal, although that might seem counter-intuitive.

\item{Cluster-aware random work-stealing}.
The word ``cluster'' refers to a group of computers that are connected
to each other by means of a fast local network.
In turn, clusters can be connected to each other, but usually the
network between clusters is much slower, and/or has a much higher latency.
The cluster-aware random work-stealing strategy
is very similar to random work-stealing, except that
each participant first tries to steal jobs from other participants in its
own cluster (thus using the fast local network).
It will only go to participants from other clusters if no participant on the
same cluster has work.

To determine to which cluster a participant belongs, Satin uses the
parent of the location as obtained from the underlying Ibis IPL.
The default for this is the domain part of the hostname. However,
the location can be set using the \texttt{ibis.location} property.

\item{Master-worker}
This strategy is suitable for applications where all Satin jobs are
generated on a single participant, the ``master''. All other participants
are ``workers'', they obtain jobs from the master and execute them.
\end{description}

Making sure that all participants always can find work to do may need
some tuning. Jobs must not be too small, because otherwise the mechanism
for obtaining jobs, which may involve network traffic, is too expensive.
On the other hand, there must be enough jobs to keep everybody busy.
The \texttt{SatinObject} class has a boolean method \texttt{needMoreJobs()},
which indicates whether it would be useful to generate more jobs.

\section{Other SatinObject methods}

The \texttt{pause()} method pauses Satin's operation. When the application
contains a large sequential part, this method can be called to temporarily
pause Satin's internal load distribution strategies to avoid communication
overhead during the execution of sequential code.
To resume Satin's operation, the \texttt{resume()} method must be used.

\section{Satin program arguments}

The Satin system accepts the following parameters, which are
passed to java as system property values.
\begin{description}
\item{\texttt{-Dsatin.closed}}
Only use the initial set of hosts for the computation; do not allow
further hosts to join the computation later on.
\item{\texttt{-Dsatin.stats}}
Display some statistics at the end of the Satin run. This is the default.
\item{\texttt{-Dsatin.stats=false}}
Don't display statistics.
\item{\texttt{-Dsatin.detailedStats}}
Display detailed statistics for every member at the end of the Satin run.
\item{\texttt{-Dsatin.alg}=\emph{algorithm}}
Specify the load-balancing algorithm to use. The possible values for
\emph{algorithm} are: RS for random work-stealing, CRS for cluster-aware
random-work stealing, and MW for master-worker.
\end{description}

\section{Satin Shared Objects}

The divide-and-conquer model operates by subdividing the problem into
subproblems (subtasks) and solving them recursively.
The only way of sharing data between tasks is by passing parameters
and returning results.
Therefore, a task can share data with its subtasks and the
other way round, but the subtasks cannot share data with each other.
Therefore, we have extended the model, and in the process renamed it
"divide-and-share".

In the divide-and-share model, tasks can share data using
\textit{shared objects}.
Updates performed on a shared object are visible to all tasks.
Operations on shared objects are executed \textit{atomically}.
Satin guarantees that shared object operations do not run concurrently with
each other. Satin also guarantees that the shared object operations do not
run concurrently with divide-and-conquer tasks. An operation performed
by a task becomes visible to other tasks only when the system reaches a
so-called \textit{safe point}: when a task is creating (spawning)
subtasks, when a task is waiting for its subtasks to finish, or when a
task completes. Tasks can also explicitly poll for shared object
updates. This makes the model clean and easy to use, as the programmer
does not need to use locks and semaphores to synchronize access to
shared data.

Shared objects are replicated on every
processor taking part in the computation. Replication is implemented using
an update protocol with function shipping: \textit{write methods},
that is, methods that modify
the state of the object, are forwarded to all processors
which apply them on their local replicas. \textit{Read methods}, that
is, methods that do not change the state of the objects, are
executed locally. 

A shared object type must be marked as such by extending the special class
\texttt{ibis.satin.SharedObject}.
This class exports a single method
{\small
\begin{quote}
\begin{verbatim}
    public void exportObject();
\end{verbatim}
\end{quote}
}
which may optionally be invoked to inform Satin that it might be useful to
broadcast the object to all participants in the run. 

Write methods must be marked as such, by specifying them in an interface
that extends the marker interface \texttt{ibis.satin.WriteMethodsInterface}.
Note that it is up to the user to specify which methods are write methods.
Here is a small example:

{\small
\begin{quote}
\begin{verbatim}
interface MinInterface extends ibis.satin.WriteMethodsInterface {
    public void set(int val);
}

final class Min extends ibis.satin.SharedObject
        implements MinInterface {
    private int val = Integer.MAX_VALUE;
    public void set(int newVal) {
        if (newVal< val) val = newVal;
    }
    public int get() { return val; }
}

\end{verbatim}
\end{quote}
}

Our shared object model provides a relaxed consistency model called
\textit{guard consistency}.
Under guard consistency, the user can define the application consistency
requirements using \textit{guard functions}. Guard functions are
associated with divide-and-conquer tasks. Conceptually, a guard
function is executed before each divide-and-conquer task.
A guard checks the state of the shared objects accessed by the task and
returns \textit{true} if those objects are in a correct state,
or \textit{false} otherwise.
Satin allows replicas to become inconsistent as long as
guards are satisfied: the updates are propagated to remote replicas on a
best effort basis.
Satin does not guarantee that the updates will not be lost or duplicated.
Updates may be applied in different order on different replicas.
When a guard is unsatisfied, Satin invalidates the local replica of a shared
object and fetches a consistent replica from another processor.

For each spawnable method, the programmer may define a boolean guard
method which specifies what the state of shared object parameters should
be before actually executing the spawned method.
Such a guard function must be defined in the same class as the spawnable
method it guards.
The name of the guard function is
\texttt{guard\_}\textit{spawnable\_function}.
It must have exactly the same parameter list as the spawnable method
and return a boolean value.
Therefore, it will have access to exactly the same shared
objects and it can check their consistency.
It will also have access to other parameters of the Satin task on which the
consistency state of the objects may depend. 
Satin takes care of invoking this method at the right moment.
Here is a small example:

{\small
\begin{quote}
\begin{verbatim}
    /* A spawnable method. */
    public Result compute(int iteration, Data data) {
        ...
    }

    /* Guard: make sure that data represents the right iteration. */
    public boolean guard_compute(int iteration, Data data) {
        return data.iteration == iteration-1;
    }
\end{verbatim}
\end{quote}

\section{Further Reading}

The Javadoc included in the \texttt{javadoc} directory has detailed
information on all classes and their methods.

The Ibis web page \url{http://www.cs.vu.nl/ibis} lists all
the documentation and software available for Ibis, including papers, and
slides of presentations.

For detailed information on running a Satin application see the
User's Manual, available in the docs directory of the Satin
distribution.

\end{document}
