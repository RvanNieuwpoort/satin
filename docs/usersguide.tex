\documentclass[a4paper,10pt]{article}

%\usepackage{graphicx}
\usepackage{url}
\usepackage{times}

\begin{document}

\title{Satin Divide-and-Conquer System User's Guide}

\author{http://www.cs.vu.nl/ibis}

\maketitle

\section{Introduction}

This manual describes the steps required to run an application that
uses the Satin Divide-and-Conquer System. How to create such an application
is described in the Satin Divide-and-Conquer Programmer's Manual.
Here, we will discuss how to compile and run the example that is described
in the Programmer's Manual.

\section{Compiling the example}

The example application for the Satin Divide-and-Conquer System is
provided with the Satin distribution, in the \texttt{examples} directory.
For convenience, the application is already compiled.

If you change the example, you will need to recompile it. This
requires the build system \texttt{ant}\footnote{\url{http://ant.apache.org}}.
Running \texttt{ant} in the examples directory compiles the example.

Invoking \emph{ant clean compile} compiles a sequential version
of the application, 
leaving the class files in a directory called \texttt{tmp}.

If, for some reason, it is not convenient to use \emph{ant} to compile
your application, or you have only class files or jar files available
for parts of your application, it is also possible to first compile
your application to class files or jar files, and then process those
using the \emph{satinc} script. This script can be found in the Satin
bin directory. It takes either directories, class files, or jar files
as parameter, and processes those, possibly rewriting them. In case
of a directory, all class files and jar files in that directory or
its subdirectories are processed.  The command sequence

\begin{verbatim}
$ cd $SATIN_HOME/examples
$ mkdir tmp
$ javac -d tmp -g \
     -classpath ../lib/satin-2.1.jar:../lib/ibis-io-2.1.jar \
     src/search/*.java
$ ../bin/satinc -cp tmp \
     -satin "search.SearchImpl1 search.SearchImpl2 search.SearchImpl3" \
     tmp
\end{verbatim}

creates a \texttt{tmp} directory and stores the resulting class files there.
The \texttt{SATIN\_HOME} environment variable must be set to the location of
the Satin installation.

Note that the Satin bytecode rewriter \texttt{satinc} needs to know the
classes in your application that must be rewritten.
Those classes are: the main class, any class that implements a shared object,
and all classes that invoke spawns or syncs.
When using the \texttt{satinc} script, these classes must be provided as
a comma-separated list, with the \texttt{-satin} option. In
\texttt{build.xml} they are provided as the \texttt{satin-classes} property.

TODO running a satin application
\end{document}
