\documentclass[a4paper,10pt]{article}

%\usepackage{graphicx}
\usepackage{url}
\usepackage{times}

\begin{document}

\title{Satin Divide-and-Conquer System User's Guide}

\author{http://www.cs.vu.nl/ibis}

\maketitle

\section{Introduction}

This manual describes the steps required to run an application that
uses the Satin Divide-and-Conquer System. How to create such an application
is described in the Satin Divide-and-Conquer Programmer's Manual.
Here, we will discuss how to compile and run the example that is described
in the Programmer's Manual.

\section{Compiling the example}

The example application for the Satin Divide-and-Conquer System is
provided with the Satin distribution, in the \texttt{examples} directory.
For convenience, the application is already compiled.

If you change the example, you will need to recompile it. This
requires the build system \texttt{ant}\footnote{\url{http://ant.apache.org}}.
Running \texttt{ant} in the examples directory compiles the example.

Invoking \emph{ant clean compile} compiles a sequential version
of the application, 
leaving the class files in a directory called \texttt{tmp}.

If, for some reason, it is not convenient to use \emph{ant} to compile
your application, or you have only class files or jar files available
for parts of your application, it is also possible to first compile
your application to class files or jar files, and then process those
using the \emph{satinc} script. This script can be found in the Satin
bin directory. It takes either directories, class files, or jar files
as parameter, and processes those, possibly rewriting them. In case
of a directory, all class files and jar files in that directory or
its subdirectories are processed.  The command sequence

\begin{verbatim}
$ cd $SATIN_HOME/examples
$ mkdir tmp
$ javac -d tmp -g \
     -classpath ../lib/satin-2.1.jar \
     src/search/*.java
$ ../bin/satinc -cp tmp \
     -satin "search.SearchImpl1,search.SearchImpl2,search.SearchImpl3" \
     tmp
\end{verbatim}

creates a \texttt{tmp} directory and stores the resulting class files there.
The \texttt{SATIN\_HOME} environment variable must be set to the location of
the Satin installation.

Note that the Satin bytecode rewriter \texttt{satinc} needs to know the
classes in your application that must be rewritten.
Those classes are: the main class, any class that implements a shared object,
and all classes that invoke spawns or syncs.
When using the \texttt{satinc} script, these classes must be provided as
a comma-separated list, with the \texttt{-satin} option. In
\texttt{build.xml} they are provided as the \texttt{satin-classes} property.

\section{Running a Satin application}

Before discussing
the running of a Satin application, we will discuss services that are
needed by the Ibis communication library, on which Satin is built.

\subsection{The pool}

A central concept in Ibis is the \emph{Pool}. A pool consists of one or
more Ibis instances, usually running on different machines. Each pool is
generally made up of Ibises running a single distributed application.
Ibises in a pool can communicate with each other, and, using the
registry mechanism present in Ibis, can search for other Ibises in the
same pool, get notified of Ibises joining the pool, etc. To
coordinate Ibis pools a so-called \emph{Ibis server} is used.

\subsection{The Ibis Server}

The Ibis server is the Swiss-army-knife server of the Ibis project.
Services can be dynamically added to the server. By default, the Ibis
communication library comes with a registry service. This registry
service manages pools, possibly multiple pools at the same time.

In addition to the registry service, the server also allows
Ibises to route traffic over the server if no direct connection is
possible between two instances due to firewalls or NAT boxes. This is
done using the Smartsockets library of the Ibis project.

The Ibis server is started with the \texttt{satin-server} script which is
located in the \texttt{bin} directory of the Satin distribution.  Before
starting a Satin application, an Ibis server needs to be running on a
machine that is accessible from all nodes participating in the Satin run.
The server listens to a TCP port. The port number can be specified using
the \texttt{--port} command line option to the \texttt{ipl-server}
script.  For a complete list of all options, use the \texttt{--help}
option of the script. One useful option is the  \texttt{--events}
option, which makes the registry print out events (such as Satin instances
joining a pool).

\subsubsection{Hubs}
\label{hubs}

The Ibis server is a single point which needs to be reachable from every
Ibis instance. Since sometimes this is not possible due to firewalls,
additional \emph{hubs} can be started to route traffic, creating a
routing infrastructure for the Satin instances. These hubs can be started
by using satin-server script with the \texttt{--hub-only} option. In
addition, each hub needs to know the location of as many of the other
hubs as possible. This information can be provided by using the
\texttt{--hub-addresses} option. See the \texttt{--help} option of the
satin-server script for more information.

\subsection{Running the example}

When the Ibis server is running, the Satin application itself can be
started. 

\end{document}
