\documentclass[a4paper,10pt]{article}

%\usepackage{graphicx}
\usepackage{url}
\usepackage{times}

\begin{document}

\title{Satin Divide-and-Conquer System User's Guide}

\author{http://www.cs.vu.nl/ibis}

\maketitle

\section{Introduction}

This manual describes the steps required to run an application that
uses the Satin Divide-and-Conquer System. How to create such an application
is described in the Satin Divide-and-Conquer Programmer's Manual.
Here, we will discuss how to compile and run the example that is described
in the Programmer's Manual.

Since Satin is built on top of the Ibis Portability Layer (IPL),
the Satin release contains the Ibis communication library, which contains
implementations of the IPL. Parts of this manual may look familiar for
readers that are familiar with the Ibis communication library.

\section{Compiling the Fibonacci example}

The Fibonacci example application for the Satin Divide-and-Conquer System is
provided with the Satin distribution, in the \texttt{examples} directory.
For convenience, the application is already compiled.

If you change the example, you will need to recompile it. This
requires the build system \texttt{ant}\footnote{\url{http://ant.apache.org}}.
Running \texttt{ant} in the examples directory compiles the example,
and rewrites the class files for use with Satin.

Invoking \emph{ant clean compile} compiles a sequential version
of the application,
leaving the class files in a directory called \texttt{tmp}.

If, for some reason, it is not convenient to use \emph{ant} to compile
your application, or you have only class files or jar files available
for parts of your application, it is also possible to first compile
your application to class files or jar files, and then process those
using the \emph{satinc} script. This script can be found in the Satin
bin directory. It takes either directories, class files, or jar files
as parameter, and processes those, possibly rewriting them. In case
of a directory, all class files and jar files in that directory or
its subdirectories are processed.  The command sequence

\begin{verbatim}
$ cd $SATIN_HOME/examples
$ mkdir tmp
$ javac -d tmp -g \
     -classpath ../lib/satin-2.1.jar \
     src/fib/*.java
$ ../bin/satinc -cp tmp -satin "fib.Fib" tmp
$ mkdir lib
$ ( cd tmp ; jar c . ) > lib/satin-examples.jar
$ rm -rf tmp
\end{verbatim}

creates a \texttt{lib} directory and stores the resulting class files there,
in a jar-file called \texttt{satin-examples.jar}.
The \texttt{SATIN\_HOME} environment variable must be set to the location of
the Satin installation.

Note that the Satin bytecode rewriter \texttt{satinc} needs to know the
classes in your application that must be rewritten.
Those classes are: the main class, any class that implements a shared object,
and all classes that invoke spawns or syncs.
When using the \texttt{satinc} script, these classes must be provided as
a comma-separated list, with the \texttt{-satin} option. In
\texttt{build.xml} they are provided as the \texttt{satin-classes} property.

\section{A Satin run}

Before discussing
the running of a Satin application, we will discuss services that are
needed by the Ibis communication library.

\subsection{The pool}

A central concept in Ibis is the \emph{Pool}. A pool consists of one or
more Ibis instances, usually running on different machines. Each pool is
generally made up of Ibises running a single distributed application.
Ibises in a pool can communicate with each other, and, using the
registry mechanism present in Ibis, can search for other Ibises in the
same pool, get notified of Ibises joining the pool, etc. To
coordinate Ibis pools a so-called \emph{Ibis server} is used.

\subsection{The Ibis Server}

The Ibis server is the Swiss-army-knife server of the Ibis project.
Services can be dynamically added to the server. By default, the Ibis
communication library comes with a registry service. This registry
service manages pools, possibly multiple pools at the same time.

In addition to the registry service, the server also allows
Ibises to route traffic over the server if no direct connection is
possible between two instances due to firewalls or NAT boxes. This is
done using the Smartsockets library of the Ibis project.

The Ibis server is started with the \texttt{satin-server} script which is
located in the \texttt{bin} directory of the Satin distribution.  Before
starting a Satin application, an Ibis server needs to be running on a
machine that is accessible from all nodes participating in the Satin run.
The server listens to a TCP port. The port number can be specified using
the \texttt{--port} command line option to the \texttt{ipl-server}
script.  For a complete list of all options, use the \texttt{--help}
option of the script. One useful option is the  \texttt{--events}
option, which makes the registry print out events (such as Satin instances
joining a pool).

\subsubsection{Hubs}
\label{hubs}

The Ibis server is a single point which needs to be reachable from every
Ibis instance. Since sometimes this is not possible due to firewalls,
additional \emph{hubs} can be started to route traffic, creating a
routing infrastructure for the Satin instances. These hubs can be started
by using satin-server script with the \texttt{--hub-only} option. In
addition, each hub needs to know the location of as many of the other
hubs as possible. This information can be provided by using the
\texttt{--hub-addresses} option. See the \texttt{--help} option of the
satin-server script for more information.

\subsection{Running the example: preliminaries}

When the Ibis server is running, the Satin application itself can be
started.  There are a number of requirements that need to be met before
Ibis (and thus Satin) can be started correctly.
In this section we will discuss these in detail.

Several of the steps below require the usage of \emph{system properties}.
System properties can be set in Java using the \texttt{-D} option of the
\texttt{java} command. Be sure to use appropriate quoting for your
command interpreter.

As an alternative to using system properties, it is also possible to use
a java properties file
\footnote{\url{http://java.sun.com/j2se/1.5.0/docs/api/java/util/Properties.html}}.
A properties file is a file containing one property per line, usually of
the format \texttt{property = value}. Properties of Ibis can be set in
such a file as if they were set on the command line directly.

Ibis and Satin will look for a file named \texttt{ibis.properties} in the
current working directory, on the class path, and at a location specified
with the \texttt{ibis.properties.file} system property.

\subsubsection{Add jar files to the classpath}

The Satin implementation is provided in a single jar file: satin.jar,
appended with the version of Satin, for instance \texttt{satin-2.1.jar}.
Satin interfaces to Ibis using the Ibis Portability Layer, or
\emph{IPL}. Both Satin and the IPL depend on various other libraries.
All jar files in \$SATIN\_HOME/lib need to be on the classpath.

\subsubsection{Configure Log4j}

Ibis and Satin use the Log4J library of the Apache project to print debugging
information, warnings, and error messages. This library must be
initialized. A configuration file can be specified using the
\texttt{log4j.configuration} system property. For example, to use a file
named \texttt{log4j.properties} in the current directory, use the
following command line option:
\texttt{-Dlog4j.configuration=file:log4j.properties}. For more info,
see the log4j website \footnote{\url{http://logging.apache.org/log4j}}.

\subsubsection{Set the location of the server and hubs}

To communicate with the registry service, each Ibis instance needs the address
of the Ibis server. This address must be specified by using the
\texttt{ibis.server.address} system property. The full address needed is
printed on start up of the Ibis server.

For convenience, it is also possible to only provide an address, port number
pair, e.g. \texttt{machine.domain.com:5435} or even simply a host, e.g.
\texttt{localhost}. In this case, the default port number (8888) is implied.
The port number provided must match the one given to the Ibis server
with the \texttt{--port} option.

When additional hubs are started (see Section \ref{hubs}), their locations
must be provided to the Ibis instances. This can be done using
the \texttt{ibis.hub.addresses} property. Ibis expects a comma-separated
list of addresses of hubs. Ibis will use the first reachable hub on the
list. The address of the Ibis server is appended to this list
automatically. Thus, by default, the Ibis server itself is used as the
hub.

\subsubsection{Set the name and size of the pool}

Each Ibis instance belongs to a pool. The name of this pool must be provided
using the \texttt{ibis.pool.name} property. With the help of the Ibis server,
this name is then used to locate other Ibis instances which belong to the
same pool. Since the Ibis server can service multiple pools simultaneously,
each pool must have a unique name.

It is possible for pools to have a fixed size. In these so-called \emph{closed
world} pools, the number of Ibises in the pool is also needed to function
correctly. This size must be set using the \texttt{ibis.pool.size} property.
This property is normally not needed. When it is needed, but not provided, Ibis
will print an error. 

\subsubsection{Satin system properties}

Satin recognizes the following system properties, which can either
be provided on the command line, or in a \texttt{ibis.properties} file
as discussed above:
\begin{description}
\item{\texttt{satin.closed}}
Only use the initial set of hosts for the computation; do not allow
further hosts to join the computation later on.
\item{\texttt{satin.stats}}
Display some statistics at the end of the Satin run. This is the default.
\item{\texttt{satin.stats=false}}
Don't display statistics.
\item{\texttt{satin.detailedStats}}
Display detailed statistics for every member at the end of the Satin run.
\item{\texttt{satin.alg}=\emph{algorithm}}
Specify the load-balancing algorithm to use. The possible values for
\emph{algorithm} are: RS for random work-stealing, CRS for cluster-aware
random-work stealing, and MW for master-worker.
\item{\texttt{satin.ft.naive}}
Fault tolerance in Satin is implemented by recomputing jobs lost as a
result of processor crashes.
To improve the performance of the
algorithm, a Global Result Table (GRT) may be used, in which results of
so-called "orphan" jobs are stored, which otherwise would have to be discarded.
The results of orphan jobs are reused while recomputing jobs lost in
crashes. The \texttt{satin.ft.naive} property specifies that Satin is
to use the naive version of its fault tolerance implementation, without
a GRT. This means that sub-jobs of lost jobs will all be recomputed.
The default is that a GRT is used.
\end{description}

\subsubsection{The satin-run script}

To simplify running a Satin application, a \texttt{satin-run} script is
provided with the distribution. This script can be
used as follows

\begin{center}
\texttt{satin-run} \emph{java-flags class parameters}
\end{center}

The script performs the first two steps needed to run an application
using Satin. It adds all required jar files
to the class path, and configures log4j.
It then runs \texttt{java} with any
command line options given to it. Therefore, any additional options for
Java, the main class and any application parameters must be provided as
if \texttt{java} was called directly.

The \texttt{satin-run} script needs the location of the Satin
distribution. This must be provided using the SATIN\_HOME environment
variable.

\subsection{Running the example on Unix-like systems}

This section is specific for Unix-like systems. In particular, the
commands presented are for a Bourne shell or bash.

We will now run the example. All code below assumes the SATIN\_HOME
environment variable is set to the location of the Satin distribution.

First, we will need an Ibis server. Start a shell and
run the \texttt{satin-server} script:
\noindent
{\small
\begin{verbatim}
$ $SATIN_HOME/bin/satin-server --events
\end{verbatim}
}
\noindent

By providing the \texttt{--events} option the server
prints information on when Ibis instances join and leave the pool.

Next, we will start the application two times. One instance will act as the
"Satin master", the other one will be a "Satin client".
Satin will determine who is who automatically. Therefore we can using the same
command line for both master and client.
Run the following command in two different shells:

\noindent
{\small
\begin{verbatim}
$ CLASSPATH=$SATIN_HOME/examples/lib/satin-examples.jar \
    $SATIN_HOME/bin/satin-run \
    -Dsatin.closed -Dibis.server.address=localhost \
    -Dibis.pool.size=2 -Dibis.pool.name=test \
    fib.Fib 32
\end{verbatim}
}
\noindent

This sets the CLASSPATH environment variable to the jar file of the
application, and calls satin-run. You should now have two running
instances of your application. One of them should print:

\noindent
{\small
\begin{verbatim}
Running Fib 32
application time fib (32) took 10.787 s
application result fib (32) = 2178309
\end{verbatim}
}
\noindent

or something similar, followed by some Satin statistics.

As said, the satin-run script is only provided for convenience. To run
the application without satin-run, the command below can be used.
Note that this only works with Java 6. For Java 1.5, you need to
explicitly add all jar files in \$SATIN\_HOME/lib to the classpath.

\noindent
{\small
\begin{verbatim}
$ java \
    -cp \
    $SATIN_HOME/lib/'*':$SATIN_HOME/examples/lib/satin-examples.jar \
    -Dibis.server.address=localhost \
    -Dibis.pool.name=test -Dibis.pool.size=2 \
    -Dsatin.closed \
    -Dlog4j.configuration=file:$SATIN_HOME/log4j.properties \
    fib.Fib 32
\end{verbatim}
}
\noindent

\subsection{Running the example on Windows systems}

We will now run the example on a Windows XP system.
All code below assumes the SATIN\_HOME
environment variable is set to the location of the Satin distribution.

First, we will need an Ibis server. Start a command prompt window and
run the \texttt{satin-server} script:
\noindent
{\small
\begin{verbatim}
C:\DOCUME~1\Temp> "%SATIN_HOME%"\bin\satin-server --events
\end{verbatim}
}
\noindent

Note the quoting, which is needed when SATIN\_HOME contains spaces.

By providing the \texttt{--events} option the server
prints information on when Ibis instances join and leave the pool.

Next, we will start the application two times. One instance will act as the
"Satin master", the other one will be a "Satin client".
Satin will determine who is who automatically. Therefore we can using the same
command line for both master and client.
Run the following command in two different shells:

\noindent
{\small
\begin{verbatim}
C:\DOCUME~1\Temp> set CLASSPATH=%SATIN_HOME%\examples\lib\satin-examples.jar
C:\DOCUME~1\Temp> "%SATIN_HOME%"\bin\satin-run
    "-Dsatin.closed" "-Dibis.server.address=localhost"
    "-Dibis.pool.size=2" "-Dibis.pool.name=test"
    fib.Fib 32
\end{verbatim}
}
\noindent

This sets the CLASSPATH environment variable to the jar file of the
application, and calls satin-run. You should now have two running
instances of your application. One of them should print:

\noindent
{\small
\begin{verbatim}
Running Fib 32
application time fib (32) took 10.787 s
application result fib (32) = 2178309
\end{verbatim}
}
\noindent

or something similar, followed by some Satin statistics.

As said, the satin-run script is only provided for convenience. To run
the application without satin-run, the command below can be used.
Note that this only works with Java 6. For Java 1.5, you need to
explicitly add all jar files in \$SATIN\_HOME/lib to the classpath.

\noindent
{\small
\begin{verbatim}
C:\DOCUME~1\Temp> java
    -cp "%SATIN_HOME%\lib\*";"%SATIN_HOME%"\examples\lib\satin-examples.jar
    -Dibis.server.address=localhost
    -Dibis.pool.name=test -Dibis.pool.size=2
    -Dsatin.closed
    -Dlog4j.configuration=file:"%SATIN_HOME%"\log4j.properties
    fib.Fib 32
\end{verbatim}
}
\noindent

\section{Further Reading}

The Ibis web page \url{http://www.cs.vu.nl/ibis} lists all
the documentation and software available for Ibis, including papers, and
slides of presentations.

For detailed information on developing a Satin application see the
Satin Programmers Manual, available in the docs directory of the Satin
distribution.

\end{document}

